\documentclass{article}
\usepackage{arxiv}
\usepackage[utf8]{inputenc} % allow utf-8 input
\usepackage[T1]{fontenc}    % use 8-bit T1 fonts
\usepackage{hyperref}       % hyperlinks
\usepackage{url}            % simple URL typesetting
\usepackage{booktabs}       % professional-quality tables
\usepackage{amsfonts}       % blackboard math symbols
\usepackage{nicefrac}       % compact symbols for 1/2, etc.
\usepackage{microtype}      % microtypography
\usepackage{lipsum}

% For usage of LaTeX, see the template: https://www.overleaf.com/latex/templates/style-and-template-for-preprints-arxiv-bio-arxiv/fxsnsrzpnvwc

\title{Understanding Alpha Decay}
\author{
  Yuan-Ru Lin
   \And
  Xing-Yao Qiu
   \And
  Heng Xu
}

\begin{document}
\maketitle

\begin{abstract}
Abstract. change to see if Overleaf-GitHub integration is good, guess it would work
\end{abstract}

\section{Introduction}
We read an article which discuss various method for solving $\alpha$-decay. \cite{understandingAlphaDecay}


\bibliographystyle{unsrt}  
\begin{thebibliography}{1}
\bibitem{understandingAlphaDecay}
George Kour and Raid Saabne.
\newblock Real-time segmentation of on-line handwritten arabic script.
\newblock In {\em Frontiers in Handwriting Recognition (ICFHR), 2014 14th
  International Conference on}, pages 417--422. IEEE, 2014.

\end{thebibliography}


\end{document}
